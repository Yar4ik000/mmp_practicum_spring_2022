
\documentclass[10pt,fleqn]{article}

\usepackage[russian]{babel}
\usepackage[utf8]{inputenc}
\usepackage{color}
\usepackage{amsmath}
\usepackage{amssymb}
\usepackage{graphics}
\usepackage{epsfig}
\usepackage{bm}
\usepackage[colorlinks,urlcolor=blue]{hyperref}
\usepackage{tikz}
\usepackage{pgfplots}
\usepackage{verbatim}
\usepackage{mdframed}
\usepackage{dirtree}
\usepackage{indentfirst}
\usepackage{url}
\usepackage{float}

\usepackage[T2A]{fontenc}
\usepackage{nccmath}
\usepackage{amssymb, amsmath, mathrsfs, amsthm}
\usepackage{graphicx}
\usepackage[footnotesize]{caption2}
\usepackage{multirow}
\usepackage{fancyvrb}
\usepackage{etoolbox}
\usepackage{pgfplotstable}
\usepackage{minted}
\usepackage[normalem]{ulem}  % для зачекивания текста
\usepackage[left=15mm, top=12mm, right=15mm, bottom=12mm, nohead=10, nofoot=10]{geometry}

\definecolor{codegray}{gray}{0.9}
\newcommand{\code}[1]{%
  \begingroup\setlength{\fboxsep}{1pt}%
  \colorbox{codegray}{\texttt{\hspace*{2pt}\vphantom{Ay}#1\hspace*{2pt}}}%
  \endgroup
}

% mdinlinecode command for including code snippets inline
% (fake verbatim, so all special character should be escaped,
% or textmode equivalents of special characters should be used)
\definecolor{mdinlinecodeboxframecolor}{HTML}{DDDDDD}
\definecolor{mdinlinecodeboxbackgroundcolor}{HTML}{F8F8F8}
\newcommand{\mdinlinecode}[1]{%
    \begin{tikzpicture}[baseline=0ex]%
        \node[anchor=base,%
            text height=0.9em,%
            text depth=0.9ex,%
            inner ysep=0pt,%
            draw=mdinlinecodeboxframecolor,%
            fill=mdinlinecodeboxbackgroundcolor,%
            rounded corners=1.5pt] at (0,0) {\small\texttt{#1}};%
    \end{tikzpicture}%
}

\newmdenv[font=\footnotesize,%
linewidth=0.4pt,%
roundcorner=2pt,%
linecolor=mdinlinecodeboxframecolor,%
backgroundcolor=mdinlinecodeboxbackgroundcolor,%
settings={\setlength{\parindent}{0pt}}]{mdcdblk}
\newenvironment{mdcodeblock}{\endgraf\verbatim}{\endverbatim}
\BeforeBeginEnvironment{mdcodeblock}{\begin{mdcdblk}}
\AfterEndEnvironment{mdcodeblock}{\end{mdcdblk}}

\textheight=26cm % высота текста
\textwidth=18cm % ширина текста
\oddsidemargin=-1cm % отступ от левого края
\topmargin=-3cm % отступ от верхнего края
\sloppy

\newcounter{example}

%-- Обозначение вектора жирным символом
\def\vec#1{\mathchoice{\mbox{\boldmath$\displaystyle#1$}}
{\mbox{\boldmath$\textstyle#1$}} {\mbox{\boldmath$\scriptstyle#1$}} {\mbox{\boldmath$\scriptscriptstyle#1$}}}

\DeclareMathOperator{\B}{Bin}
\DeclareMathOperator{\Ps}{Poiss}
\DeclareMathOperator{\R}{Unif}
\DeclareMathOperator{\sign}{\mathrm{sign}}
\DeclareMathOperator{\softmax}{\mathrm{softmax}}
\DeclareMathOperator{\loss}{\mathcal{L}}

\pagestyle{empty}

\begin{document}

\begin{center}
    \begin{tabular}{|p{17.5cm}|}
        \hline
        \textbf{ВМК}\\
        \begin{center} \Large Задание 5. MapReduce. Коллаборативная фильтрация. \end{center}\\
        \textbf{Практикум 317 группы, весна 2021-2022}\\
        \hline
    \end{tabular}
\end{center}

\

\begin{tabbing}
    Начало выполнения задания: 28 апреля 2022 года, 02:00.\\
    Жёсткий Дедлайн: \textcolor{red}{\bf 23 мая 2022 года, 23:59.}
\end{tabbing}

%\tableofcontents


\begin{section}{Введение. Рекомендательные системы}
Сегодня рекомендательные системы встречаются повсеместно. В интернет-магазине Вы можете увидеть блоки с «похожими товарами», на новостном сайте «похожие новости» или «новости, которые могут вас заинтересовать», на сайте с арендой фильмов это могут быть блоки с «похожими фильмами» или «рекомендуем вам посмотреть».

Задача рекомендательной системы заключается в нахождении небольшого числа фильмов (\mdinlinecode{Item}), которые
скорее всего заинтересуют конкретного пользователя (\mdinlinecode{User}), используя информацию о предыдущей его активности и характеристиках фильмов.

Широко известен конкурс компании \mdinlinecode{Netflix}, которая в 2006 году предложила предсказать оценки пользователя для фильмов в шкале от $1$ до $5$ по известной части оценок. Победителем признавалась команда, которая улучшит \mdinlinecode{RMSE} на тестовой выборке на $10\%$ по сравнению с их внутренним решением. За время проведения конкурса появилось много новых методов решения поставленной задачи.

Обычно в таких задачах выборка представляет собой тройки $(u, i, r_{u,i})$, где $u$ – пользователь, $i$ – фильм, $r_{u,i}$ – рейтинг. Далее будем считать, что рейтинги нормализованы на отрезке $[0, 1]$.

\begin{subsection}{Neighborhood подход в коллаборативной фильтрации}
    Имея матрицу user-item из оценок пользователей можно определить меру \mdinlinecode{adjusted cosine similarity} похожести товаров $i$ и $j$ как векторов в пространстве пользователей:
    \begin{ceqn}
    \begin{equation} \label{eq:1}
    sim(i, j) = \frac{\sum\limits_{u \in U_{i,j}}(r_{u, i} - \overline{r_{u}})(r_{u,j} - \overline{r_{u}})}{\sqrt{\sum\limits_{u \in U_{i,j}}(r_{u, i} - \overline{r_{u}})^{2}}\sqrt{\sum\limits_{u \in U_{i,j}}(r_{u, j} - \overline{r_{u}})^{2}}}
    \end{equation}
    \end{ceqn}
    где $U_{i,j}$ – множество пользователей, которые оценили фильмы $i$ и $j$, $\overline{r_{u}}$ – средний рейтинг пользователя $u$.

    Рейтинги для неизвестных фильмов считаются по следующей формуле:
    \begin{ceqn}
    \begin{equation} \label{eq:2}
    \hat{r}_{u,i}=\frac{\sum\limits_{j \in I_{u}}sim(i,j)r_{u,j}}{\sum\limits_{j \in I_{u}}sim(i,j)}
    \end{equation}
    \end{ceqn}
    где $I_{u}$ - множество фильмов, которые оценил пользователь $u$. Такой подход называется \mdinlinecode{item-oriented}. Обратим внимание на то, что $sim(i, j) \in [-1, 1]$. Это может привести к делению на ноль или значениям $\hat{r}_{u,i}$ вне диапазона $[0, 1]$. Избавимся от этой проблемы, положив равными нулю отрицательные значения $sim(i, j)$.
\end{subsection}
\end{section}

\begin{section}{Описание задания}
В рамках данного задания Вам будет необходимо реализовать коллаборативную фильтрацию по формулам \ref{eq:1}, \ref{eq:2} с использованием фреймворка \mdinlinecode{MapReduce}. Ваша программа, получая на вход список троек $(u, i, r_{u,i})$ и список соответствий между номером фильма и его названием, должна вывести для каждого пользователя \textbf{топ-100} фильмов с самым высоким предсказанным рейтингом.

При вычислений рекомендаций необходимо учитывать только те фильмы, которые пользователь ещё не оценил. Рекомендации выводятся по убыванию предсказанной оценки. При равенстве предсказанных оценок выше в списке рекомендаций должен стоять фильм с лексикографически меньшим названием.
\newpage
Строки в файле с предсказаниями необходимо представить в следующем виде:
\begin{figure}[H]
    \begin{center}
        \textbf{<user\_id>@<rating\_1>\%<film\_name\_1>@...@<rating\_100>\%<film\_name\_100>}\\
    \end{center}
    \caption{\label{fig:out_format}Формат выходных данных}
\end{figure}

В качестве датасета предлагается использовать \href{https://grouplens.org/datasets/movielens/latest/}{MovieLens}. Используйте \mdinlinecode{small} версию датасета. Результат работы Вашего решения на этом датасете нужно приложить при сдаче задания (папка \mdinlinecode{data/output/final}).

При выполнении задания необходимо привести подробное описание Вашего решения в файле \mdinlinecode{description.md/html/pdf}. В частности:
\begin{enumerate}
	\item Описание каждой стадии выполнения программы и каждой map-reduce задачи
	\item Сложность по числу операций и по количеству памяти для каждого маппера и редьюсера. Используйте следующие обозначения: $U$ -- общее число пользователей, $I$ -- общее число фильмов, $M$ -- число мапперов, $R$ -- число редьюсеров,
	$\alpha$ -- средняя доля фильмов, оценённых одним пользователем (эквивалентно средней доле пользователей, оценивших один фильм и доле известных оценок к общему числу возможных оценок $UI$)
	\item Суммарное время работы вашей программы
	\item Решение бонусных заданий
\end{enumerate}

\end{section}

\begin{section}{Технические детали реализации}
Обратите внимание на следующие моменты, которые помогут успешно решить задачу:
\begin{itemize}
	\item \href{https://github.com/nakhodnov17/docker-hadoop-spark/tree/master/examples/wordcount_streaming}{Пример запуска Hadoop Streaming программы на кластере}
	\item Для использования пользовательских сепараторов используйте следующие опции:
	\item[]
		\begin{minted}{bash}
		-D stream.num.map.output.key.fields=<number_of_fields_for_key>
		-D stream.map.output.field.separator=<custom_separator>
		-D stream.reduce.input.field.separator=<custom_separator>
		-D mapreduce.map.output.key.field.separator=<custom_separator>
		\end{minted}
	\item Для реализации вторичной сортировки могут пригодится следующие опции:
	\item[]
		\begin{minted}{bash}
		-D mapreduce.partition.keycomparator.options=<sort_options>
		-D mapreduce.job.output.key.comparator.class= \
				org.apache.hadoop.mapred.lib.KeyFieldBasedComparator
		-D mapreduce.partition.keypartitioner.options=<partition_options>
		-partitioner org.apache.hadoop.mapred.lib.KeyFieldBasedPartitioner
		\end{minted}
	\item Для управления памятью, выделяемой контейнерами используйте следующие опции:
	\item[]
		\begin{minted}{bash}
		-D mapreduce.map.memory.mb=<physical_memory_for_each_mapper>
		-D mapreduce.reduce.memory.mb=<physical_memory_for_each_reducer>
		-D mapreduce.map.java.opts=-Xmx<heap_size_for_each_mapper>m
		-D mapreduce.reduce.java.opts=-Xmx<heap_size_for_each_reducer>m
		-D yarn.nodemanager.vmem-pmem-ratio=<virtual_to_physical_memory_ratio>
		\end{minted}
	\item[] Подробнее смотрите в \href{https://stepik.org/lesson/21001/step/7?unit=5085}{курсе на Stepik}
	\item Детали этих опций и другие параметры смотрите в \href{https://hadoop.apache.org/docs/current/hadoop-streaming/HadoopStreaming.html}{документации по Hadoop Streaming}
	\item Учтите, что $sim(i, i) = 1 \;\forall i \in \{1,...,I\}$. Также, гарантируйте отсутствие деления на ноль при вычислениях по формулам \ref{eq:1}, \ref{eq:2}, добавив к знаменателю небольшое число (например, \mdinlinecode{$10^{-9}$})
	\item Используйте ровно \mdinlinecode{$3$} знака после запятой для передачи вещественных чисел между воркерами
	\item В файле \mdinlinecode{movies.csv} названия фильмов могут содержать запятые, поэтому используйте \mdinlinecode{csv.reader} из библиотеки \mdinlinecode{csv} для корректного разбиения строк этого файла.
	\item Задание можно реализовать так, чтобы время выполнение задачи было \mdinlinecode{$\approx 30$ минут}. Если Ваша программа выполняется сильно дольше, значит Ваша реализация неоптимальна. Неоптимальные реализации будут существенно штрафоваться
\end{itemize}

\end{section}

\begin{section}{Формат сдачи задания}
При выполнении задания необходимо использовать docker-контейнеры для запуска Hadoop кластера. Инструкцию по развертыванию контейнеров смотрите \href{https://github.com/nakhodnov17/docker-hadoop-spark}{в репозитории}. Другие форматы сдачи допускаются только  по согласованию с преподавателем.

\begin{figure}[H]
    \begin{center}
        \begin{minipage}[b]{0.8\textwidth}
            \renewcommand*\DTstylecomment{\color{blue}}
            \renewcommand*\DTstyle{\ttfamily\textcolor{red}}
            \dirtree{%
            .1 collaborative\_filtering.
            .2 data.
            .3 input\DTcomment{Входные данные для MapReduce задачи}.
            .4 rating.csv.
            .4 movies.csv.
            .3 output\DTcomment{Результат всех MapReduce задач}.
            .4 stage\_1.
            .5 \_SUCCESS.
            .5 part-00000.
            .5 ....
            .4 ....
            .4 final\DTcomment{Итоговые рекомендации}.
            .5 \_SUCCESS.
            .5 part-00000.
            .5 ....
            .2 src\DTcomment{Исходники MapReduce алгоритма}.
            .3 mapper\_1.py.
            .3 reducer\_1.py.
            .3 ....
            .3 mapper\_n.py.
            .3 reducer\_n.py.
            .2 description.md\DTcomment{Описание решения}.
            .2 run\_hadoop.sh\DTcomment{Скрипт для запуска Hadoop Streaming на namenode}.
            .2 run.sh\DTcomment{Скрипт запуска MR задачи}.
            }
        \end{minipage}
    \end{center}
    \caption{\label{fig:proj_struct}Требуемая структура решения}
\end{figure}

В результате, проект должен в точности удовлетворять структуре на диаграмме \ref{fig:proj_struct}. В качестве решения необходимо сдать архив данной структуры с названием \mdinlinecode{<ФИО>\_task\_05.zip}.

Некоторые детали формата сдачи:
\begin{itemize}
    \item Скрипт \mdinlinecode{run.sh}, обеспечивает полную работу задачи, начиная от загрузки данных внутрь docker-контейнера и запуска \mdinlinecode{run\_hadoop.sh}, заканчивая копированием результата работы из docker-контейнера. Запуск программы должен подразумевать только запуск этого скрипта
	\item Файл \mdinlinecode{description.md} содержит описание Вашего решения. Все формулы в описании должны быть оформлены в виде \LaTeX-уравнений. Допускается использовать \mdinlinecode{Markdown} синтаксис для оформления решения. Также, можете использовать \mdinlinecode{Jupyter Notebook}, сохранённый в формате \mdinlinecode{HTML}
	\item Папка \mdinlinecode{src} содержит код, используемый в задании. В частности, для каждой map-reduce стадии Вашей программы должны быть файлы \mdinlinecode{mapper\_<stage\_n>.py}, \mdinlinecode{reducer\_<stage\_n>.py}. Если используются другие скрипты (комбайнер и так далее) они также должны иметь в названии номер соответствующей стадии
	\item Папка \mdinlinecode{data/input}, содержит входные данные (файлы \mdinlinecode{ratings.csv}, \mdinlinecode{movies.csv}). При проверке решения входные файлы будут располагаться в этой директории
    \item В папке \mdinlinecode{data/output}, после завершения работы \mdinlinecode{run.sh} должны содержаться результаты работы программы. В частности, внутри директории \mdinlinecode{data/output.stage\_<stage\_n>} дожен находиться выход редьюсера (или маппера в случае map-only задачи) соответствующей стадии. В папке \mdinlinecode{final} должен быть итоговый список рекомендаций. Также, Вы можете предусмотреть сохранение любых промежуточных результатов в папке \mdinlinecode{data/output}. Присылать сами промежуточные файлы не требуется, необходимо прислать только содержимое папки \mdinlinecode{data/output/final}
    \item Разрешена обработка данных вне HDFS для дальнейшего их использования в Replicated Join
    \item Использование map-reduce стадий только с одним редьюсером запрещено
    \item Решение будет оценено в полный балл только при условии, что запуск \mdinlinecode{run.sh} на разных входных данных будет успешен.

\end{itemize}

\end{section}

\begin{section}{Бонусные задания}
\subsection{Анализ полученного решения (3 балла)}
Выполните следующие пункты:
\begin{enumerate}
	\item Реализуйте скрипт для генерации данных, похожих на настоящие. Реализуйте генерацию для различных $U, I, \alpha$
	\item Замерьте время работы каждой стадии программы для различных значений $U, I, \alpha, M, R$. Обязательно сделайте замеры времени для датасета большего объема, чем в датасете \mdinlinecode{small}
	\item Постройте графики зависимости времени работы от разных параметров
	\item Докажите, используя полученные данные, что асимптотика Вашего решения соответствует теоретической (например, можно отдельно нарисовать графики в \mdinlinecode{log-log} шкале). Если наблюдается несоответствие, то объясните почему
\end{enumerate}
\subsection{Использование фреймворка (2 балла)}
Реализуйте Вашу программу с использованием фреймворка \href{https://mrjob.readthedocs.io/en/latest/}{mrjob}.
\subsection{Запуск на больших данных (2 балла)}
Запустите задачу на \href{https://grouplens.org/datasets/movielens/10m/}{существенно больших данных}. Обратите внимание, выполнение может занять порядка \sout{6-7}14-15 часов и до 70Gb памяти. Засеките время работы каждой стадии и сравните с временем работы на \mdinlinecode{small} датасете. Согласуются ли полученные результаты с теоретическими формулами для сложности алгоритма? Если нет, то почему?
\end{section}

\end{document}
