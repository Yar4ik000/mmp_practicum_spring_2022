
\documentclass[10pt,fleqn]{article}

\usepackage[russian]{babel}
\usepackage[utf8]{inputenc}
\usepackage{color}
\usepackage{amsmath}
\usepackage{amssymb}
\usepackage{graphics}
\usepackage{epsfig}
\usepackage{bm}
\usepackage[colorlinks,urlcolor=blue]{hyperref}
\usepackage{tikz}
\usepackage{pgfplots}
\usepackage{verbatim}
\usepackage{mdframed}
\usepackage{dirtree}
\usepackage{indentfirst}
\usepackage{url}
\usepackage{float}

\definecolor{codegray}{gray}{0.9}
\newcommand{\code}[1]{%
  \begingroup\setlength{\fboxsep}{1pt}%
  \colorbox{codegray}{\texttt{\hspace*{2pt}\vphantom{Ay}#1\hspace*{2pt}}}%
  \endgroup
}

% mdinlinecode command for including code snippets inline
% (fake verbatim, so all special character should be escaped,
% or textmode equivalents of special characters should be used)
\definecolor{mdinlinecodeboxframecolor}{HTML}{DDDDDD}
\definecolor{mdinlinecodeboxbackgroundcolor}{HTML}{F8F8F8}
\newcommand{\mdinlinecode}[1]{%
    \begin{tikzpicture}[baseline=0ex]%
        \node[anchor=base,%
            text height=0.9em,%
            text depth=0.9ex,%
            inner ysep=0pt,%
            draw=mdinlinecodeboxframecolor,%
            fill=mdinlinecodeboxbackgroundcolor,%
            rounded corners=1.5pt] at (0,0) {\small\texttt{#1}};%
    \end{tikzpicture}%
}

\newmdenv[font=\footnotesize,%
linewidth=0.4pt,%
roundcorner=2pt,%
linecolor=mdinlinecodeboxframecolor,%
backgroundcolor=mdinlinecodeboxbackgroundcolor,%
settings={\setlength{\parindent}{0pt}}]{mdcdblk}
\newenvironment{mdcodeblock}{\endgraf\verbatim}{\endverbatim}
\BeforeBeginEnvironment{mdcodeblock}{\begin{mdcdblk}}
\AfterEndEnvironment{mdcodeblock}{\end{mdcdblk}}

\textheight=26cm % высота текста
\textwidth=18cm % ширина текста
\oddsidemargin=-1cm % отступ от левого края
\topmargin=-3cm % отступ от верхнего края
\sloppy

\newcounter{example}

%-- Обозначение вектора жирным символом
\def\vec#1{\mathchoice{\mbox{\boldmath$\displaystyle#1$}}
{\mbox{\boldmath$\textstyle#1$}} {\mbox{\boldmath$\scriptstyle#1$}} {\mbox{\boldmath$\scriptscriptstyle#1$}}}

\DeclareMathOperator{\B}{Bin}
\DeclareMathOperator{\Ps}{Poiss}
\DeclareMathOperator{\R}{Unif}
\DeclareMathOperator{\sign}{\mathrm{sign}}
\DeclareMathOperator{\softmax}{\mathrm{softmax}}
\DeclareMathOperator{\loss}{\mathcal{L}}

\pagestyle{empty}

\begin{document}

\begin{center}
    \begin{tabular}{|p{17.5cm}|}
        \hline
        \textbf{ВМК}\\
        \begin{center} \Large Задание 4. Hadoop. HDFS. MapReduce. \end{center}\\
        \textbf{Практикум 317 группы, весна 2021-2022}\\
        \hline
    \end{tabular}
\end{center}

\

\begin{tabbing}
    Начало выполнения задания: 13 апреля 2022 года, 12:00.\\
    Мягкий Дедлайн: \textcolor{blue}{\bf 27 апреля 2022 года, 23:59.} \\
    Жёсткий Дедлайн: \textcolor{red}{\bf 04 мая 2022 года, 23:59.}
\end{tabbing}

%\tableofcontents

\section*{Формулировка задания}
Данное задание направлено на знакомство с инфраструктурой Apache Hadoop.


Задание состоит из двух частей:
\begin{enumerate}
 \item Выполнение базовых действий в файловой системе HDFS.
 \item Реализация и запуск простейшей MapReduce программы.
\end{enumerate}

\begin{section}{Базовые действия в HDFS (1 балл)}
    В данном задании необходимо написать скрипт \mdinlinecode{hdfs.sh}, выполняющий следующую последовательность действий из корневой директории \mdinlinecode{namenode} Hadoop кластера:
    \begin{enumerate}
        \item Создать локально файл \mdinlinecode{test.txt} размером $100$Mb
        \item Создать hdfs-директории \mdinlinecode{temp} и \mdinlinecode{logs}
        \item Записать файл \mdinlinecode{test.txt} в директорию \mdinlinecode{temp}
        \item Посмотреть свойства записанного файла
        \item Переместить файл \mdinlinecode{test.txt} в директорию \mdinlinecode{logs}
        \item Установить фактор репликации для файла равным $1$
        \item Скопировать \mdinlinecode{test.txt} в \mdinlinecode{test2.txt}
        \item Скопировать директорию \mdinlinecode{logs} в \mdinlinecode{logs2} с помощью \mdinlinecode{hadoop distcp}
        \item Установить права доступа, означающие чтение и запись только для владельца для файла \mdinlinecode{test2.txt} в директории \mdinlinecode{logs2}
        \item Вывести свойства всех файлов в \mdinlinecode{logs2}
        \item Посмотреть размер всех директорий в \mdinlinecode{/}
        \item Удалить директорию \mdinlinecode{logs}
        \item Запустить \mdinlinecode{fsck} на директории \mdinlinecode{logs2}
        \item Посмотреть отчет о HDFS через \mdinlinecode{dfsadmin}
        \item Переместить \mdinlinecode{/logs2/test2.txt} в локальную папку \mdinlinecode{/}
        \item Добавить содержимое локального файла \mdinlinecode{test2.txt} в конец файла \mdinlinecode{/logs2/test.txt} в \mdinlinecode{hdfs}
        \item Вывести размер для каждого файла в \mdinlinecode{/logs2} в Mb
    \end{enumerate}

Каждый пункт должен соответствовать одной команде. Результат выполнения (stdout, stderr) \mdinlinecode{hdfs.sh} сохраните в файле \mdinlinecode{hdfs.output.txt}.
\newpage
Также, ответьте на следующие вопросы в файле \mdinlinecode{hdfs.answers.md}:
\begin{enumerate}
\item Сколько сплитов было обработано командой \mdinlinecode{hadoop distcp}? Сколько мапперов и сколько редьюсеров было задействовано в процессе её выполнения?
\item Сколько реплик блоков отсутствует после выполнения команды \mdinlinecode{fsck}? Объясните в чём причина отсутствия реплик.
\item Какой размер файловой системы HDFS?
\end{enumerate}

В качестве решения необходимо сдать три файла \mdinlinecode{hdfs.sh}, \mdinlinecode{hdfs.output.txt}, \mdinlinecode{hdfs.answers.md} с в точности такими именами.
\end{section}

\begin{section}{MapReduce умножение матриц (4 балла)}
    В данном пункте необходимо реализовать программу для Hadoop Streaming, реализующую умножение матриц.

    Положим $A \in \mathbb{R}^{n \times m}, B \in \mathbb{R}^{m \times k}, C = AB \in \mathbb{R}^{n \times k}$.

    Будем представлять входные данные (то есть матрицы $A$, $B$) в виде файлов \mdinlinecode{A.txt}, \mdinlinecode{B.txt}, где элементы матриц записаны построчно через пробел. При этом, явно пронумеруем все строки. Результат перемножения (\mdinlinecode{С.txt}) будем представлять в виде, пар ключ-значение, где в качестве ключа выступает пара номер строки, номер столбца. Порядок строк при этом будем игнорировать.
    $$
    A = \begin{pmatrix}
    3 & 4 & 3 & 2 \\
    0 & 5 & 2 & 1 \\
    0 & 4 & 2 & 0
    \end{pmatrix}  \Longrightarrow
    \underbrace{
    \begin{tabular}{ccccc}
        0 & 3 & 4 & 3 & 2 \\
        1 & 0 & 5 & 2 & 1 \\
        2 & 0 & 4 & 2 & 0
    \end{tabular}
    }_\text{A.txt} \quad\quad\quad\quad\quad\quad
    C = \begin{pmatrix}
    1 & 2 \\
    3 & 4 \\
    \end{pmatrix}  \Longrightarrow
    \underbrace{
    \begin{tabular}{cc}
        0,1 2 \\
        0,0 1 \\
        1,0 3 \\
        1,1 4
    \end{tabular}
    }_\text{C.txt}
    $$

    Вам необходимо реализовать MapReduce программу, которая принимает на вход матрицы $A,B$ в описанном выше формате, вычисляет их произведение и возвращает результат в виде списка ключей-значений.

    В задании предоставляются некоторые вспомогательные скрипты, которые упростят решение задачи:
    \begin{enumerate}
        \item \mdinlinecode{generate\_task.py} -- генерирует случайные матрицы $A,B$ заданного размера вычисляет их произведение и сохраняет $A,B,C$ в необходимом формате.

        Пример запуска: \mdinlinecode{python3 generate\_task.py -n 10 -m 20 -k 30}
        \item \mdinlinecode{check\_answer.py} -- сравнивает истинный ответ \mdinlinecode{С.txt} с ответом, сгенерированным MapReduce программой. Скрипт учитывает, что результат MapReduce программы разбит на файлы с соответствии с числом редьюсеров.

        Пример запуска: \mdinlinecode{python3 check\_answer.py -n 10 -m 20 -k 30}
    \end{enumerate}

    Для решения задачи необходимо реализовать следующие скрипты:
    \begin{enumerate}
        \item \mdinlinecode{mapper.py}, \mdinlinecode{reducer.py} -- код маппера и редьюсера для перемножения матриц
        \item \mdinlinecode{run\_hadoop.sh} -- код для подготовки директорий на namenode, запуска Hadoop Streaming задачи, и копирования результата из HDFS в локальную файловую систему namenode
        \item \mdinlinecode{run.sh} -- скрипт для запуска всего пайплайна задачи. Данный скрипт должен принимать в качестве аргументов командной строки размерность задачи ($n,m,k$), число мапперов и число редьюсеров, с которым будет запускаться MapReduce задача на кластере.

        Сначала, выполняется генерация матриц $A,B,C$ заданного размера, затем происходит копирование данных внутрь контейнера и запуск Hadoop Streaming задачи на namenode. По завершении MapReduce программы, скрипт копирует результат из контейнера и проверяет его на корректность

        Пример запуска: \mdinlinecode{sh run.sh 10 20 30 5 5}
    \end{enumerate}

    В результате, проект должен в точности удовлетворять структуре на диграмме \ref{fig:proj_struct}. Решение будет оценено в полный балл только при условии, что запуск \mdinlinecode{run.sh} с различными параметрами будет успешен.
\begin{figure}[H]
    \begin{center}
        \begin{minipage}[b]{0.8\textwidth}
            \renewcommand*\DTstylecomment{\color{blue}}
            \renewcommand*\DTstyle{\ttfamily\textcolor{red}}
            \dirtree{%
            .1 matmul\_streaming.
            .2 data.
            .3 input\DTcomment{Входные данные для MapReduce задачи}.
            .4 A.txt.
            .4 B.txt.
            .3 output\DTcomment{Результат выполнения MapReduce задачи}.
            .4 \_SUCCESS.
            .4 part-00000.
            .4 ....
            .3 C.txt\DTcomment{Правильный ответ на задачу}.
            .2 src\DTcomment{Исходники MapReduce алгоритма}.
            .3 mapper.py.
            .3 reducer.py.
            .2 check\_answer.py\DTcomment{Скрипт для проверки решения на корректность}.
            .2 generate\_task.py\DTcomment{Скрипт для генерации данных для задачи умножения матриц}.
            .2 run\_hadoop.sh\DTcomment{Скрипт для запуска Hadoop Streaming на namenode}.
            .2 run.sh\DTcomment{Скрипт для генерации MR задачи, её запуска и проверки}.
            }
        \end{minipage}
    \end{center}
    \caption{\label{fig:proj_struct}Требуемая структура решения}
\end{figure}

Несколько важных замечаний, которые помогут при решении:
\begin{enumerate}
    \item За основу решения можно взять \href{https://github.com/nakhodnov17/docker-hadoop-spark/tree/master/examples/wordcount_streaming}{решение задачи Word Count}. В частности, обратите внимание на скрипт \mdinlinecode{run\_hadoop.sh}
    \item Для передачи размерности задачи внутрь маппера и редьюсера можно использовать переменные окружения. Подробнее про установку переменных окружения для воркеров можно прочитать \href{https://hadoop.apache.org/docs/current/hadoop-streaming/HadoopStreaming.html#Setting_Environment_Variables}{здесь}
    \item Флаги для изменения числа мапперов, редьюсеров и информацию о других тонких настройках Hadoop Streaming можно найти в \href{https://hadoop.apache.org/docs/current/hadoop-streaming/HadoopStreaming.html}{документации}
    \item Для решения задачи Вам может потребоваться определить, какой именно файл обрабатывается воркером в данный момент. Для решения этой задачи можно также использовать \href{https://github.com/nakhodnov17/docker-hadoop-spark/blob/a7672eb8b7b4ddcc8b6871353e0e14fb706c68db/examples/wordcount_streaming/src/mapper.py#L8}{переменные окружения}
    \item При выводе и сохранении чисел с плавающей точкой обратите внимание на число знаков после запятой. Рекомендуется использовать порядка $5$ знаков после запятой, для достижения приемлемой точности. Учтите, что большое количество знаков после запятой кратно замедляет работу MapReduce приложений из-за увеличения объёма передаваемых данных
    \item Для копирования данных в контейнера и из контейнера можно использовать специальные \href{https://docs.docker.com/engine/reference/commandline/cp/}{команды docker}
    \item Для запуска программ внутри контейнера также есть специальные \href{https://docs.docker.com/engine/reference/commandline/exec/}{команды docker}
\end{enumerate}

\end{section}

\end{document}
